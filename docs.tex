\documentclass{article}
    % General document formatting
    \usepackage[margin=0.7in]{geometry}
    \usepackage[parfill]{parskip}
    \usepackage[utf8]{inputenc}

    \usepackage{amsmath,amssymb,amsfonts,amsthm, mathtools}
    \usepackage{listings,xcolor,caption}

    \newcommand{\divides}{\mid}
    \newcommand{\notdivides}{\nmid}

    \definecolor{codegreen}{rgb}{0,0.6,0}
    \definecolor{codegray}{rgb}{0.5,0.5,0.5}
    \definecolor{codepurple}{rgb}{0.58,0,0.82}
    \definecolor{backcolour}{rgb}{0.95,0.95,0.92}

    \lstdefinestyle{mystyle} {
        backgroundcolor=\color{backcolour},
        commentstyle=\color{codegreen},
        keywordstyle=\color{magenta},
        numberstyle=\tiny\color{codegray},
        stringstyle=\color{codepurple},
        basicstyle=\ttfamily\footnotesize,
        breakatwhitespace=false,
        breaklines=true,
        captionpos=b,
        keepspaces=true,
        numbers=left,
        numbersep=5pt,
        showspaces=false,
        showstringspaces=false,
        showtabs=false,
        tabsize=2,
        escapeinside={tex(}{tex)}
    }
    \lstset{style=mystyle}
    \renewcommand{\lstlistingname}{Algoritmo}

    \usepackage{mathtools}
    \DeclarePairedDelimiter{\ceil}{\lceil}{\rceil}

\begin{document}

\section{Análises de Complexidade}

\subsection{Exponenciação Binária}
\label{binExp}

O algoritmo de exponenciação binária é utilizado para calcular $a^b \bmod p$, com $a, b, p \in \mathbb{Z}$. O algoritmo utiliza da seguinte recorrência:

\begin{align*}
  a^b \bmod p = \begin{cases}
    1 & \text{se } b = 0 \\
    {(a^{b/2})}^2 & \text{se } b \bmod 2 = 0 \\
    a \cdot {(a^{(b-1)/2})}^2 & \text{se } b \bmod 2 = 1 \\
\end{cases}
\end{align*}

A cada 2 passos da recorrência, o valor de $b$ diminui, pelo menos, pela metade. A recorrência para quando $b = 0$ e as transições são todas constantes, logo a complexidade de tempo do algoritmo é $O(\log_2 b)$.

\begin{minipage}{.9\linewidth}
\begin{lstlisting}[language=haskell,caption=Exponenciação Binária]
binExp :: Integer -> Integer -> Integer -> Integer
binExp a b p = f' 1 a b where
    f' r a b
        | b == 0 = r
        | odd b = f' (r*a `mod` p) (a^2 `mod` p) (shiftR b 1)
        | otherwise = f' r (a^2 `mod` p) (shiftR b 1)

\end{lstlisting}
\end{minipage}

\subsection{Algoritmo Extendido de Euclides}

O algoritmo extendido de Euclides calcula, para entradas $a, b \in \mathbb{Z}$, a tripla $(g, x, y) \in \mathbb{Z} \times \mathbb{Z} \times \mathbb{Z}$ tal que:

\begin{align*}
  g & = \text{mdc}(a, b) \\
  g & = a \cdot x + b \cdot y \\
\end{align*}

Para isso, ele utiliza a seguinte recorrência

\begin{align*}
  \text{emdc}(a, b) = \begin{cases}
    (a, 1, 0) & \text{se } b = 0 \\
    (g, y, x - y \cdot q) & \text{c.c. }, \text{ onde } q \cdot a + r = b, (g, x, y) = \text{emdc}(b, r) \\
\end{cases}
\end{align*}

A cada dois passos da recorrência, o maior valor diminui pelo menos pela metade, pois $\forall a, b \in \mathbb{N} \colon a \geq b \implies a \bmod b \leq \frac{a}{2}$. Logo, a complexidade total do algoritmo é $O(\log_2(\max (a, b)))$.

Em particular, o Algoritmo Extendido de Euclides é utilizado para calcular o \textbf{Inverso Modular} de um inteiro $a \bmod m$. Para isso, basta calcular o $x$ tal que $ax + my = \text{mdc(a, m)}$ já que se $a$ é inversível, $\text{mdc}(a, m) = 1$ e $x \bmod m$ é seu inverso modular.


\begin{minipage}{.9\linewidth}
\begin{lstlisting}[language=haskell,caption=Algoritmo Extendido de Euclides]
egcd :: Integer -> Integer -> (Integer, Integer, Integer)
egcd a 0 = (a, 1, 0)
egcd a b =
    let (g, x1, y1) = egcd b r
    in (g, y1, x1 - y1 * q)
        where
            (q, r) = divMod a b
\end{lstlisting}
\end{minipage} \\
Observe que a implementação para encontrar o inverso modular de um número é apenas uma chamada ao algoritmo extendido de Euclides.

\begin{minipage}{.9\linewidth}
\begin{lstlisting}[language=haskell,caption=Inverso Modular]
invMod :: Integer -> Integer -> Integer
invMod a m =
    let (g, x, y) = egcd a m
    in case g of
        1 -> (x `mod` m + m) `mod` m
        _ -> error $ "invMod expects coprime numbers, received " ++ show a ++ " " ++ show m
\end{lstlisting}
\end{minipage}

\subsection{Teorema Chinês do Resto}

Para calcular a solução de um sistema de equações modulares com módulos coprimos:

\begin{align*}
  x \equiv \begin{cases}
    &a_1 \bmod m_1 \\
    &a_2 \bmod m_2 \\
    &\vdots \\
    &a_k \bmod m_k \\
\end{cases}
\end{align*}

utilizamos a fórmula fechada do \textit{Teorema Chinês do Resto}, onde $M_i$ é o produto de todos os módulos menos o i-ésimo e $N_i \coloneq M_i^{-1} \bmod m_i$:

\[
  x \equiv \sum_{i=1}^{k} a_iM_iN_i (\bmod m_1m_2 \ldots m_k)
\]

Para isso, é calculado o produtório de todos os primos em $O(k)$, extraídos cada um dos $M_i, N_i$ em $O(\log_2 M_i) \approx O(\log_2 \prod_i m_i) = O(k \cdot \log_2 m)$ e por fim são somados todos os $a_iM_iN_i$ em $O(k)$, resultando em uma complexidade final de $O(k + k*(k* \log_2 m) + k) = O(k^2 \log_2 m)$.

\begin{minipage}{.9\linewidth}
\begin{lstlisting}[language=haskell,caption=Teorema Chinês do Resto]
crt :: [(Integer, Integer)] -> Integer
crt equations =
    let
        prod = product $ map snd equations
        res = sum $ map (\(a, n) ->
            let
                p = prod `div` n
                inv = invMod p n
            in a * p * inv) equations
    in res `mod` prod
\end{lstlisting}
\end{minipage}

\subsection{Crivo de Eratóstenes}

Para pré-calcular uma lista de primos pequenos, utilizamos do algoritmo do \textbf{Crivo de Eratóstenes}. Em particular, utilizamos uma versão com um consumo menor de memória, o crivo \textit{segmentado}, mas o princípio e complexidade são as mesmas da versão clássica e ela será explicada aqui.

Inicializamos a lista de primos com o $2$ e um vetor de booleanos, assumindo a priori que todos os números são primos, menos o $1$. Em seguida, percorremos os números de $3$ até um certo limite $N$, com passos de $+2$ pois nenhum número par diferente de 2 é primo. Por fim, toda vez que chegamos em um número checamos se ele já foi marcado como não primo. Se sim, continuamos, se não adicionamos ele na lista e marcamos todos os seus múltiplos como não primos. Esse algoritmo tem complexidade $O(\sum_{p \leq n} \frac{n}{p}) = O(n \cdot \sum_{p \leq n}\frac{1}{p})$. Não será demonstrado aqui, mas isso é $O(n \cdot \log \log n)$.

\begin{minipage}{0.9\linewidth}
\begin{lstlisting}[language=haskell,caption=Crivo de Eratóstenes e Lista dos primos]
-- Crivo Segmentado
sieveSeg a' b' ps = [i*2+1 | i <- [a..b], coprime ! i] where
a = shiftR a' 1; b = shiftR b' 1
muls p = [l, l+p .. b] where l = shiftR (p * ((a'+p-1) `div` p .|. 1)) 1
coprime = accumArray (\_ b -> b) True (a, b) (map (,False) $ concatMap muls ps) :: UArray Integer Bool

-- Lista dos Primos
primes :: [Integer] = [2,3] ++ sieve 5 where
sieve n = sieveSeg n top ps ++ sieve (top+2) where
    top = min (n + 2^15) (2 + n*n - 4*n)
    ps = takeWhile (\p -> p*p < top) (tail primes)
\end{lstlisting}
\end{minipage}

Observe que aqui definimos um limite para o crivo, ou seja, o $n$-ésimo primo que é calculado em tempo hábil:

\begin{minipage}{0.9\linewidth}
\begin{lstlisting}[language=haskell,caption=Limite do Crivo]
softLimit :: Integer = 2^20
\end{lstlisting}
\end{minipage}

\subsection{Teste de Primalidade}
\label{isPrime}

Para o teste de primalidade, temos três casos:

\begin{enumerate}
\item Se o número é menor que o limite da nossa lista de primos ao quadrado, checamos se ele possui algum fator na nossa lista, pois todo número não primo possui um divisor menor ou igual a sua raiz. Com isso, decidimos de forma determinística se ele é primo ou não.
\item Se ele é maior que esse limite, ainda assim checamos se ele é divisível por algum dos primos na lista. Se for, temos certeza que ele não é primo. Se ele não for, passamos para o 3 passo.
\item Rodamos o \textit{Miller Rabin} usando os 50 primeiros primos como testemunhas.
\end{enumerate}

As complexidades para os passos 1 e 2 são $O(L)$, onde $L$ é o limite da lista, pois apenas percorremos ela. Para o Miller Rabin, primeiro retiramos todas as potências de 2 de $n - 1$ em $O(\log_2 n)$, fatorando ele em $2^sd$, $d$ ímpar. Em seguida, para cada uma das testemunhas $a$, calculamos $a^d$ em $O(\log_2 d)$ com a \ref{binExp} e $s$ quadrados de $a^d$ em $O(s)$. Por fim, testamos os valores dessas potências $\bmod n$. Caso $a^d \not\equiv 1 \bmod n$ ou $a^{2^rd} \not\equiv -1 \bmod n$ para algum $r < s$, temos que $n$ não é primo. Logo, a complexidade final do algoritmo é $O(\log_2 n + 50 \cdot (\log_2 d + s)) = O(\log_2n + \log_2 d + \log_2 n) = O(\log_2 n)$.

\noindent\hspace{0.03\linewidth}
\begin{minipage}{0.9\linewidth}
\begin{lstlisting}[language=haskell,caption=Miller-Rabin]
isPrime :: Integer -> Bool
isPrime n
    | n < (softLimit^2) = isCoprime n (takeWhile (\p -> p*p <= n) primes)
    | not $ isCoprime n (take 1000 primes) = False
    | otherwise = (not . any witness) (take 50 primes) where
    isCoprime n ps = not $ any (\p -> mod n p == 0) ps
    (s, d) = f' (0, n-1) where
        f' (s, n)
            | even n = f' (s+1, shiftR n 1)
            | otherwise = (s, n)
    witness a =
        let
            x = binExp a d n
            squares = iterate (\x -> x^2 `mod` n) x
        in all (\x -> x /= 1 && x /= n-1) (take (s+1) squares)
\end{lstlisting}
\end{minipage}

\subsection{Primeiro Primo Maior que N}

De acordo com o \textit{Postulado de Bertrand}, pra todo $n$, o próximo primo maior que $n$ é menor que $2n$, logo precisamos testar a primalidade de apenas $O(n)$ números com o algoritmo \ref{isPrime}, resultando em um algoritmo $O(n\cdot (\log_2 n))$.

Além disso, utilizamos de \textit{wheel factorization} para iterar apenas por números coprimos e assim reduzir o número de candidatos testados.


\noindent\hspace{0.03\linewidth}
\begin{minipage}{0.9\linewidth}
\begin{lstlisting}[language=haskell,caption=Wheel Factorization e Primeiro Primo Maior que N]
-- Roda de fatoracao
factWheel :: Int -> [Integer]
factWheel k = sieveSeg (n+1) (n+n-1) ps where
    ps = tail $ take k primes
    n = 2 * product ps

wheel5 :: [Integer] = factWheel 5
wheel5Cycle :: Integer = product $ take 5 primes

-- Primeiro primo maior que n
firstPrimeGT :: Integer -> Integer
firstPrimeGT n
    | n <= wheel5Cycle = head $ dropWhile (<= n) primes
    | otherwise = let
        x = n - (n `mod` wheel5Cycle)
        wheel = [y+x-wheel5Cycle | y <- wheel5] ++ map (+wheel5Cycle) wheel
        candidates = dropWhile (<= n) wheel
    in head [x | x <- candidates, isPrime x]

\end{lstlisting}
\end{minipage}

\subsection{Pollard Rho}
\label{pollardRho}

O algoritmo de \textit{Pollard Rho} encontra um fator de um inteiro composto $n$ encontrando um ciclo na sequência $(x_i = f(x_{i-1})) \bmod p$, onde $f$ é um polinômio da forma $f(x) = x^2 + k$, para algum $p$ divisor de $n$.
Como não sabemos a priori o valor de $p$, procuramos esse ciclo através da sequência $x_i = f(x_{i-1}) \bmod n$.
Encontramos um ciclo na sequência $\bmod p$ quando achamos $x_i, x_j$ na sequência $\bmod n$ tais que $\text{mdc}(x_s - s_t, n) > 1$. Caso $\text{mdc}(x_s - x_t, n) = n$, precisamos repetir com um polinômio e valor inicial diferentes, caso contrário temos um fator de $n$. É esperado que esse ciclo seja encontrado em $O(\sqrt{p})$, pelo \textit{Paradoxo do Aniversário}.

Em particular, nossa implementação previamente testa vários valores de $k$ em $f(x) = x^2+ k$ até encontrar algum ciclo onde o \textit{mdc} seja diferente de $n$. Para controlar o tamanho da busca em cada sequência e o número de sequências testadas, usamos do valor esperado para o encontro do ciclo caso $p$ seja menor que $\sqrt{n}$, que é $O(\sqrt[4]{n})$. Caso não seja encontrado nenhum fator após todos esses testes, retornamos $1$. Utilizamos de uma constante para limitar o número de sequências testadas e cada sequência é procurada até $\min (4 \cdot \ceil{\sqrt[4]{n}}, C)$, onde $C$ é o maior valor de um inteiro em \textit{Haskell}. Logo, para números suficientemente grandes, o algoritmo tem complexidade constante.

\noindent\hspace{0.03\linewidth}
\begin{minipage}{0.9\linewidth}
\begin{lstlisting}[language=haskell,caption=Pollard Rho]
-- Encontra fator com pollardRho
pollardRho :: Integer -> Maybe Integer -> Integer
pollardRho n lim
    | isPrime n = n
    | otherwise = fromMaybe 1 ds where
        cap = toInteger (maxBound :: Int)
        exp = fromInteger.min cap.ceiling.(*4).sqrt.sqrt.fromIntegral$n
        max = maybe maxBound (fromInteger.min cap) lim
        f k x = mod (x*x+k) n
        xs = concatMap (\k -> take exp $ iterate (f k) 2) [1..]
        ys = concatMap (\k -> take exp $ iterate (f k.f k) 2) [1..]
        ds = find (\d -> d /= 1 && d /= n)
            . take max
            $ zipWith (\x y -> gcd n (x-y)) xs ys
\end{lstlisting}
\end{minipage}

\subsection{Fatoração}

A fatoração dos números em seus fatores primos foi implementada da seguinte forma:

\begin{enumerate}
\item Os fatores do número que estão presentes no crivo são retirados do número. Se o que resta é $1$, o algoritmo encerra.
\item Caso contrário, o resto não fatorado do número é fatorado utilizando do algoritmo~\ref{pollardRho} com um limite constante no número de iterações. Ao final do processo, possivelmente sobra um fator composto grande do número, que é retornado junto com a lista de fatores primos fatorados.
\end{enumerate}

Como as iterações do \textit{Pollard Rho} e os fatores presentes no crivo são constantes, o algoritmo no pior caso tem complexidade de tempo constante.

\noindent\hspace{0.03\linewidth}
\begin{minipage}{0.9\linewidth}
\begin{lstlisting}[language=haskell,caption=Fatoração]

-- Encontra fatores primos que estao no crivo, caso tenha
trialDiv :: Integer -> [Integer] -> ([Integer], [Integer], Integer)
trialDiv n (p:ps)
    | n == 1 = ([], [], 1)
    | p*p > n = ([n], [], 1)
    | mod n p == 0 = (p:fs', ps', r)
    | p < softLimit = trialDiv n ps
    | isPrime n = ([n], [], 1)
    | otherwise = ([], ps, n)
    where (fs', ps', r) = trialDiv (div n p) (p:ps)

-- Fatora um numero usando Pollard Rho para os fatores maiores
factorizeRho n ps lim
    | n == 1 || p == 1 = []
    | otherwise = sort $ factorizeDiv p ps lim ++ factorizeDiv (div n p) ps lim
    where p = pollardRho n lim

-- Retira os fatores do crivo e fatora o resto com Pollard Rho, dentro do limite estabelecido
factorizeDiv n ps lim
    | r == 1 = fs
    | otherwise = sort $ fs ++ factorizeRho r ps' lim
    where (fs, ps', r) = trialDiv n ps

\end{lstlisting}
\end{minipage}

Dessa forma, a primeira função não estabelece um limite de iterações (sempre encontra a fatoração total mas em tempo não viável dependendo do número), ao contrário da segunda, que retorna a fatoração parcial quando não é viável encontrar a fatoração total.  

\noindent\hspace{0.03\linewidth}
\begin{minipage}{0.9\linewidth}
\begin{lstlisting}[language=haskell,caption=Fatoração]
data Factorization = Full [Integer] | Partial [Integer] Integer
instance Show Factorization where
    show (Full fs) = intercalate "*" (map show fs)
    show (Partial fs r) = intercalate "*" (map show $ fs ++ [r]) ++ " (fatoracao parcial)"


-- Tenta fatorar o numero sem o limite de iteracoes
factorize :: Integer -> [Integer]
factorize n = sort $ factorizeDiv n primes Nothing

-- Realiza a fatoracao de um numero, com um limite de iteracoes no Pollard Rho
factorizePartial :: Integer -> Factorization
factorizePartial n = if n == p then Full fs else Partial fs (div n p) where
    p = product fs
    fs = sort $ factorizeDiv n primes (Just$2^20)
\end{lstlisting}
\end{minipage}

\subsection{Algoritmos para encontrar um Gerador/Elemento de Ordem Alta}
Foram usados dois algoritmos para encontrar um gerador(ou, elemento de ordem alta) para o grupo $\mathbb{Z}/(p\mathbb{Z})$. Vale mencionar que esses algoritmos precisam da fatoração de $p-1$, e quando isso é inviável, conseguimos encontrar apenas um elemento de ordem alta em relação à fatoração parcial de $p-1$.

\subsubsection{Algoritmo I}
Esse algoritmo se baseia na ideia inicial: testar para geradores candidatos entre $2$ e $p-1$ e cada teste se baseia em verificar se meu candidato $x$ quando elevado à $\frac{p-1}{q_i}$ é diferente de $1$ para todo $q_i$ fator potência de primo de $p-1$.
Ou seja:
\begin{align*}
    x &\in \mathbb{Z}/(p\mathbb{Z})\\
    p - 1 &= \prod_{i=1}^{m}p_i^{e_i}\\
    q_i &= p_i^{e_i}\\
    x \text{ é gerador } &\iff x^{\frac{p-1}{q_i}} \neq 1 \text{ para todo } q_i
\end{align*}

Dessa forma, cada teste tem complexidade $O(m)$, no qual $m$ é o número de fatores primos distintos de $p-1$. O menor gerador de um grupo foi provado por Grosswald(1981) ser $g_p < p^{0.499}$. Como o algoritmo realiza os testes enquanto não achar um gerador, a complexidade é $O(m \cdot p ^{0.499})$. Porém o algoritmo é bem mais rápido na prática, uma vez que é relativamente comum se ter geradores pequenos para esses grupos, dado que já se tem a fatoração de $p-1$.

\noindent\hspace{0.03\linewidth}
\begin{minipage}{.9\linewidth}
\begin{lstlisting}[language=haskell, caption=Busca pelo Menor Gerador]
smallHighOrderElement :: Factorization -> (Integer, OrderBounds)
smallHighOrderElement f = (g, o) where
    p = defactorize f + 1
    g = head $ filter isHighOrder [2..(p-1)]
    (fs, o) = case f of
        Full fs -> (fs, Exact (p-1))
        Partial fs r -> (r:fs, Bounded (product fs * firstPrimeGT softLimit) (p-1))
    isHighOrder g = all
        ((\f -> binExp g (div (p - 1) f) p /= 1) . head) (group . sort $ fs)

\end{lstlisting}
\end{minipage}

\subsubsection{Algoritmo II}
Considere que a fatoração parcial de $p-1$ é dada por $p-1 = p_1^{e_1}\cdot p_2^{e_2} \dots p_m^{e_m} \cdot N$.
Esse algoritmo se baseia na afirmação:
\begin{align*}
g_1, g_2 &\in G\\
ord(g_1) = n_1 &\text{ e } ord(g_2) = n_2\\
mdc(n_1, n_2) = 1 &\implies ord(g_1 \cdot g_2) = n_1 \cdot n_2
\end{align*}

Vamos contruir elementos de ordem específica, fator de $p-1$ da seguinte forma:
\begin{align*}
b_j &= a^{(p-1)/p_j^{e_j}} \text{ para } j = 1, 2, \dots, m\\
ord(b_j) &| p_j^{e_j}                           &&\text{Por construção}\\
b_j ^ {(p-1)/p_j} &\neq 1 \implies ord(b_j)=p_j^{e_j}
\end{align*}
Dessa forma, quando temos a fatoração total de $p-1$, para cada fator primo $p_i^{e_i}$, podemos encontrar um elemento da ordem do fator e os multiplicar para obter um gerador de ordem $p-1$. 
Já quando temos apenas a fatoração parcial, fazemos o mesmo processo para os fatores $p_i^{e_i}$ conhecidos e para o fator desconhecido $N$, porém aqui não saberemos a ordem exata para este elemento, apenas que está entre $\frac{p-1}{N}$ e $p-1$. 

\noindent\hspace{0.03\linewidth}
\begin{minipage}{.9\linewidth}
\begin{lstlisting}[language=haskell,caption=Algoritmo de Gauss]
highOrderElement :: Factorization -> (Integer, OrderBounds)
highOrderElement f = (g, o) where
    p = defactorize f + 1
    (fs, o) = case f of
        Full fs -> (fs, Exact (p-1))
        Partial fs r -> (r:fs, Bounded (product fs * firstPrimeGT softLimit) (p-1))
    g = foldl (\a b -> mod (a*b) p) 1 bs
    bs = map findB.multiplicity$fs
    findB (f, k) = head
        . filter (\b -> binExp b (div (p-1) f) p /= 1)
        . map (\a -> binExp a (div (p-1) (f^k)) p)
        $ primes
    multiplicity = map (\f -> (head f, length f)).group.sort

\end{lstlisting}
\end{minipage}

\subsection{Pohlig-Hellman Generalizado}
$$ a^k \cong b \bmod p \text{ com } G = \langle a \rangle, |G| = n, b \in G$$
$$ n = \prod_{i=1}^{m}p_i^{e_i} \text{ fatoração em primos} $$
Para cada fator primo $p_i$ com multiplicidade $e_i$, nós calculamos
\begin{align*}
    a_i &= a^{n/p_i^{e_i}}\\
    b_i &= b^{n/p_i^{e_i}}\\
        &= a^{k n/p_i^{e_i}} = a_i^k\\
    b_i = a_i^k &\implies b_i \in \langle a_i \rangle            &&\text{Por construção}\\
    ord(a_i) = p_i^{e_i} &\implies b_i = a_i^{k \bmod p_i^{e_i}} &&\text{Pelo teorema de Lagrange}
\end{align*}
Chamemos $ k_i = k \bmod p_i^{e_i} $, podemos então calcular $k_i$ em grupos de ordem $p_i^{e_i}$ ao resolver $a_i^{k_i} \equiv b_i \bmod p_i^{e_i}$, e obter $k$ ao solucionar o sistema
\begin{align*}
    k &= k_1 \bmod p_1^{e_1} \\
    k &= k_2 \bmod p_2^{e_2} \\
    &\;\;\vdots \\
    k &= k_m \bmod p_m^{e_m}
\end{align*}
A implementação calcula $k_i$ para cada fator primo de $n$ com auxílio da função Pohlig-Hellman para grupos de ordem primária em $O(e_i\sqrt{p_i})$, então calcula $k$ utilizando o algorítmo para o teorema Chinês do resto em $O(m)$.

\noindent\hspace{0.03\linewidth}
\begin{minipage}{.9\linewidth}
\begin{lstlisting}[language=haskell,caption=Pohlig Hellman]
pohligHellman :: Integer -> Integer -> [Integer] -> Maybe Integer
pohligHellman a g f
    | any (\(x, val) -> isNothing x) equations = Nothing
    | otherwise = Just $ crt (map (\(Just x, val) -> (x, val)) equations)
    where
        m = product f + 1
        groupOrder = m - 1
        facts = map (\x -> (head x, toInteger . length $ x)) . group . factorize $ groupOrder
        equations = map (\(q, e) ->
            let
                di = binExp q e m
                a' = binExp a (groupOrder `div` di) m
                g' = binExp g (groupOrder `div` di) m
                ni = pohligPrimePower a' g' (q, e) m
            in (ni, di)) facts
\end{lstlisting}
\end{minipage}


\subsection{Pohlig-Hellman com Ordem Primária}
O algorítmo de Pohlig-Hellman em um grupo com ordem primária, \emph{i.e.} potência de primo, calcula
$$ a^k = b \text{ em } G = \langle a \rangle\text{, com }|G| = p^e\text{ e }b \in G$$
Suponha que tenhamos grupos denotados por
\begin{align*}
    G_0 &= G\\
    G_i &= \lbrace x^{p^i} \vert\ x \in G \rbrace\\
    G_e &= \lbrace x^{p^e} = 1 \vert\ x \in G \rbrace   &&\text{Pela definição de ordem}\\
  |G_i| &= p^{e-i}                                      &&\text{Pelo teorema de Lagrange}
\end{align*}
Façamos uma sequência $b$
\begin{align*}
    b_0 &= b\\
    b_{i+1} &= b_i a^{-k_i p^{i}}
\end{align*}
Definimos $k_i$ como a solução para uma equação utilizando $\pi_i(x) : G_i \rightarrow G_{e-1} |\ x \in G_i$
\begin{align*}
    \pi_i(x) &= x^{p^{e-1-i}}\\
    \pi_0(a)^{k_i} &= \pi_i(b_i)
\end{align*}
Observe que $b_i \in G_i$, e portanto $b_e \in G_e$, então $b_e = 1$. Se $b_i$ está em $G_i$ então há solução $k_i$
\begin{align*}
    b_0 &\in G_0\\
    \pi_i(b_{i+1}) &= \pi_i(b_i)\pi_i(a)^{-k_ip^i}          &&\text{$\pi_i$ aplicado à definição de $b_{i+1}$}\\
    \pi_i(b_{i+1}) &= \pi_0(a)^{k_i}\pi_0(a)^{-k_i} = 1     &&\text{Usando a definição de $k_i$}\\
    b_{i+1}^{p^{e-1-i}} = 1 &\implies b_{i+1} = x^{p^{i+1}} &&\text{Pela definição de ordem}\\
    b_{i+1} &\in G_{i+1}                                    &&\text{Pela definição de $G_i$}
\end{align*}
Ao expandir $b_e$, obtemos o inverso de $b_0$
\begin{align*}
    1 &= b_e\\
    &= b_{e-1}a^{-k_{e-1} p^{e-1}}\\
    &= b_{e-2}a^{-\sum_{i=e-2}^{e-1}{k_i p^i}}\\
    &\;\;\vdots\\
    &= b_{0}a^{-\sum_{i=0}^{e-1}{k_i p^i}}\\
    a^{\sum_{i=0}^{e-1}{k_i p^i}} &= b\\
  \text{ então } k &= \sum_{i=0}^{e-1}{k_i p^i}\\
\end{align*}
Assim, resolvemos $e$ equações $\pi(a)^{k_i} = \pi(b_i)$ em $G_{e-1}$, de ordem $p$, utilizando o Baby-Step Giant-Step em $O(\sqrt{p})$, e obtemos $k$ pela equação acima. A complexidade do algorítmo é $O(e\sqrt{p})$

\noindent\hspace{0.03\linewidth}
\begin{minipage}{.9\linewidth}
\begin{lstlisting}[language=haskell,caption=Pohlig Prime Power]
pohligPrimePower :: Integer -> Integer -> (Integer, Integer) -> Integer -> Maybe Integer
pohligPrimePower b a (p, e) m = k where
    tex($\pi$tex) i x = binExp x (p^(e-1-i)) m
    solve b i = babyGiantSteps' (tex($\pi$tex) i b) (tex($\pi$tex) 0 a) m p
    it (i, b, k) = (i+1, b',) <$> solve b' (i+1) where
        b' = mod (b * binExp a (m-1-k*(p^i)) m) m
    k = fmap (sum.map (\(i,_,k) -> k*p^i))
        . sequence
        . take (fromInteger e)
        $ iterate (it=<<) ((0,b,) <$> solve b 0)
\end{lstlisting}
\end{minipage}

\subsection{Baby-Step Giant-Step}
O algoritmo Baby-Step Giant-Step é utilizado para resolver a equação
$$ a^k \cong b \bmod p \text{ com } G = \langle a \rangle, |G| = n, b \in G$$
$$ r = \ceil{\sqrt{n}}$$
Dado que $ |G| = n $, então $ a^k = a^{k \bmod n} $, e $ 0 \leq k < n $. Assim, re-escrevemos $ k = j r + i $ com $ 0 \leq i, j < r $.\\
Para encontrar $i$ e $j$, o algoritmo calcula duas sequências
\begin{align*}
    x_i &= a^i\\
    y_j &= b a^{-j r} = a^{k - j r}\\
        &= a^{j r + i - j r}\\
        &= a^{i}
\end{align*}
A implementação requer que encontremos $i$ e $j$ tal que $x_i = y_j$. As sequências tem tamanho $r$, são ordenadas em $O(r \log r)$, e buscamos elementos iguais em $O(r)$. A complexidade é $O(r \log r)$

\noindent\hspace{0.03\linewidth}
\begin{minipage}{.9\linewidth}
\begin{lstlisting}[language=haskell,caption=Baby-Steps Giant-Steps]
babyGiantSteps' :: Integer -> Integer -> Integer -> Integer -> Maybe Integer
babyGiantSteps' b a p n = f xs ys where
    r = toInteger . ceiling . sqrt . fromIntegral $ n
    s = invMod (binExp a r p) p
    xs = sortOn fst $ zip (iterate ((`mod`p).(*a)) 1) [0..r-1]
    ys = sortOn fst $ zip (iterate ((`mod`p).(*s)) b) [0..r-1]
    f [] _ = Nothing
    f _ [] = Nothing
    f xxs@((x,i):xs) yys@((y,j):ys)
        | x>y = f xxs ys
        | x<y = f xs yys
        | otherwise = Just $ j*r + i
\end{lstlisting}
\end{minipage}

\end{document}
