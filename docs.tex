\documentclass{article}
    % General document formatting
    \usepackage[margin=0.7in]{geometry}
    \usepackage[parfill]{parskip}
    \usepackage[utf8]{inputenc}

    \usepackage{amsmath,amssymb,amsfonts,amsthm}
    \usepackage{listings,xcolor,caption}

    \newcommand{\divides}{\mid}
    \newcommand{\notdivides}{\nmid}

    \definecolor{codegreen}{rgb}{0,0.6,0}
    \definecolor{codegray}{rgb}{0.5,0.5,0.5}
    \definecolor{codepurple}{rgb}{0.58,0,0.82}
    \definecolor{backcolour}{rgb}{0.95,0.95,0.92}

    \lstdefinestyle{mystyle} {
        backgroundcolor=\color{backcolour},
        commentstyle=\color{codegreen},
        keywordstyle=\color{magenta},
        numberstyle=\tiny\color{codegray},
        stringstyle=\color{codepurple},
        basicstyle=\ttfamily\footnotesize,
        breakatwhitespace=false,
        breaklines=true,
        captionpos=b,
        keepspaces=true,
        numbers=left,
        numbersep=5pt,
        showspaces=false,
        showstringspaces=false,
        showtabs=false,
        tabsize=2
    }
    \lstset{style=mystyle}
    \renewcommand{\lstlistingname}{Algorítmo}

\begin{document}

\section{Análises de Complexidade}

\subsection{Exponenciação Binária}

O algoritmo de exponenciação binária é utilizado para calcular $a^b \mod p$, com $a, b, p \in \mathbb{Z}$. O algoritmo utiliza da seguinte recorrência:

\begin{align*}
  a^b \mod p = \begin{cases}
    1 & \text{se } b = 0 \\
    {(a^{b/2})}^2 & \text{se } b \mod 2 = 0 \\
    a \cdot {(a^{(b-1)/2})}^2 & \text{se } b \mod 2 = 1 \\
\end{cases}
\end{align*}

A cada 2 passos da recorrência, o valor de $b$ diminui, pelo menos, pela metade. A recorrência para quando $b = 0$ e as transições são todas constantes, logo a complexidade de tempo do algoritmo é $O(\log_2 b)$.

\subsection{Algoritmo Extendido de Euclides}

O algoritmo extendido de Euclides calcula, para entradas $a, b \in \mathbb{Z}$, a tripla $(g, x, y) \in \mathbb{Z} \times \mathbb{Z} \times \mathbb{Z}$ tal que:

\begin{align*}
  g & = \text{mdc}(a, b) \\
  g & = a \cdot x + b \cdot y \\
\end{align*}

Para isso, ele utiliza a seguinte recorrência

\begin{align*}
  \text{emdc}(a, b) = \begin{cases}
    (a, 1, 0) & \text{se } b = 0 \\
    (g, y, x - y \cdot q) & \text{c.c. }, \text{ onde } q \cdot a + r = b, (g, x, y) = \text{emdc}(b, r) \\
\end{cases}
\end{align*}

A cada dois passos da recorrência, o maior valor diminui pelo menos pela metade, pois $\forall a, b \in \mathbb{N} \colon a \geq b \implies a \mod b \leq \frac{a}{2}$. Logo, a complexidade total do algoritmo é $O(\log_2(\max (a, b)))$.

\subsection{Pohlig-Hellman Generalizado}
O algorítmo de Pohlig-Hellman é utilizado para resolver a equação
$$ a^k \cong b \mod p \text{ com } G = \langle a \rangle, |G| = n, b \in G$$
$$ n = \prod_{i=1}^{m}p_i^{e_i} \text{ fatoração em primos} $$
Para cada fator primo $p_i$ com multiplicidade $e_i$, nós calculamos
\begin{align*}
    a_i &= a^{n/p_i^{e_i}}\\
    b_i &= b^{n/p_i^{e_i}}\\
        &= a^{k n/p_i^{e_i}} = a_i^k\\
    b_i = a_i^k &\implies b_i \in \langle a_i \rangle            &&\text{Por construção}\\
    ord(a_i) = p_i^{e_i} &\implies b_i = a_i^{k \bmod p_i^{e_i}} &&\text{Pelo teorema de Lagrange}
\end{align*}
Chamemos $ k_i = k \bmod p_i^{e_i} $, podemos então calcular $k_i$ em grupos de ordem $p_i^{e_i}$ ao resolver $a_i^{k_i} \equiv b_i \bmod p_i^{e_i}$, e obter $k$ ao solucionar o sistema
\begin{align*}
    k &= k_1 \mod p_1^{e_1} \\
    k &= k_2 \mod p_2^{e_2} \\
    &\;\;\vdots \\
    k &= k_m \mod p_m^{e_m}
\end{align*}
Que pode ser feito utilizando o algorítmo para o teorema Chinês do resto em $O(m)$

\subsection{Pohlig-Hellman com Ordem Potência de Primo}

\subsection{Baby-Step Giant-Step}
O algorítmo Baby-Step Giant-Step é utilizado para resolver a equação
$$ a^k \cong b \bmod p \text{ com } G = \langle a \rangle, |G| = n, b \in G$$
$$ r = \lceil \sqrt{n} \rceil $$
Dado que $ |G| = n $, então $ a^k = a^{k \mod n} $, e $ 0 \leq k < n $. Assim, re-escrevemos $ k = j r + i $ com $ 0 \leq i, j < r $.\\
Para encontrar $i$ e $j$, o algorítmo calcula duas sequências
\begin{align*}
    x_i &= a^i \bmod p\\
    y_j &= b a^{-j r} = a^{k - j r}\\
        &= a^{j r + i - j r}\\
        &= a^{i} = x_i \bmod p
\end{align*}

A implementação requer que encontremos $i$ e $j$ tal que $x_i = y_j$. As sequências tem tamanho $r$, são ordenadas em $O(r \log r)$, e buscamos elementos iguais em $O(r)$. A complexidade é $O(r \log r)$

\noindent\hspace{0.03\linewidth}
\begin{minipage}{.9\linewidth}
\begin{lstlisting}[language=haskell,caption=Baby-Steps Giant-Steps]
bsgs :: Integer -> Integer -> Integer -> Integer -> Maybe Integer
bsgs b a p n = f xs ys where
    r = toInteger . ceiling . sqrt . fromIntegral $ n
    s = invMod (binExp a r p) p
    xs = sortOn fst $ zip (iterate ((`mod`p).(*a)) 1) [0..r-1]
    ys = sortOn fst $ zip (iterate ((`mod`p).(*s)) b) [0..r-1]
    f [] _ = Nothing
    f _ [] = Nothing
    f xxs@((x,i):xs) yys@((y,j):ys)
        | x>y = f xxs ys
        | x<y = f xs yys
        | otherwise = Just $ j*r + i
\end{lstlisting}
\end{minipage}

\end{document}