\documentclass{article}
    % General document formatting
    \usepackage[margin=0.7in]{geometry}
    \usepackage[parfill]{parskip}
    \usepackage[utf8]{inputenc}

    \usepackage{amsmath,amssymb,amsfonts,amsthm}
    \usepackage{listings,xcolor,caption}

    \newcommand{\divides}{\mid}
    \newcommand{\notdivides}{\nmid}

    \definecolor{codegreen}{rgb}{0,0.6,0}
    \definecolor{codegray}{rgb}{0.5,0.5,0.5}
    \definecolor{codepurple}{rgb}{0.58,0,0.82}
    \definecolor{backcolour}{rgb}{0.95,0.95,0.92}

    \lstdefinestyle{mystyle} {
        backgroundcolor=\color{backcolour},
        commentstyle=\color{codegreen},
        keywordstyle=\color{magenta},
        numberstyle=\tiny\color{codegray},
        stringstyle=\color{codepurple},
        basicstyle=\ttfamily\footnotesize,
        breakatwhitespace=false,
        breaklines=true,
        captionpos=b,
        keepspaces=true,
        numbers=left,
        numbersep=5pt,
        showspaces=false,
        showstringspaces=false,
        showtabs=false,
        tabsize=2,
        escapeinside={tex(}{tex)}
    }
    \lstset{style=mystyle}
    \renewcommand{\lstlistingname}{Algoritmo}

\begin{document}

\section{Análises de Complexidade}

\subsection{Algoritmos para encontrar um Gerador/Elemento de Ordem Alta}
Foram usados dois algoritmos para encontrar um gerador(ou, elemento de ordem alta) para o grupo $\mathbb{Z}/(p\mathbb{Z})$. Vale mencionar que esses algoritmos precisam da fatoração de $p-1$, e quando isso é inviável, conseguimos encontrar apenas um elemento de ordem alta em relação à fatoração parcial de $p-1$.

\subsubsection{Algoritmo I}
Esse algoritmo se baseia na ideia inicial: testar para geradores candidatos entre $2$ e $p-1$ e cada teste se baseia em verificar se meu candidato $x$ quando elevado à $\frac{p-1}{q_i}$ é diferente de $1$ para todo $q_i$ fator potência de primo de $p-1$.
Ou seja:
\begin{align*}
    x &\in \mathbb{Z}/(p\mathbb{Z})\\
    p - 1 &= \prod_{i=1}^{m}p_i^{e_i}\\
    q_i &= p_i^{e_i}\\
    x \text{ é gerador } &\iff x^{\frac{p-1}{q_i}} \neq 1 \text{ para todo } q_i
\end{align*}

Dessa forma, cada teste tem complexidade $O(m)$, no qual $m$ é o número de fatores primos distintos de $p-1$. O menor gerador de um grupo foi provado por Grosswald(1981) ser $g_p < p^{0.499}$. Como o algoritmo realiza os testes enquanto não achar um gerador, a complexidade é $O(m \cdot p ^{0.499})$. Porém o algoritmo é bem mais rápido na prática, uma vez que é relativamente comum se ter geradores pequenos para esses grupos, dado que já se tem a fatoração de $p-1$.

\noindent\hspace{0.03\linewidth}
\begin{minipage}{.9\linewidth}
\begin{lstlisting}[language=haskell, caption=Busca pelo Menor Gerador]
smallHighOrderElement :: Factorization -> (Integer, OrderBounds)
smallHighOrderElement f = (g, o) where
    p = defactorize f + 1
    g = head $ filter isHighOrder [2..(p-1)]
    (fs, o) = case f of
        Full fs -> (fs, Exact (p-1))
        Partial fs r -> (r:fs, Bounded (product fs * firstPrimeGT softLimit) (p-1))
    isHighOrder g = all
        ((\f -> binExp g (div (p - 1) f) p /= 1) . head) (group . sort $ fs)
    
\end{lstlisting}
\end{minipage}

\subsubsection{Algoritmo II}
Considere que a fatoração parcial de $p-1$ é dada por $p-1 = p_1^{e_1}\cdot p_2^{e_2} \dots p_m^{e_m} \cdot N$.
Esse algoritmo se baseia na afirmação:
\begin{align*}
g_1, g_2 &\in G\\
ord(g_1) = n_1 &\text{ e } ord(g_2) = n_2\\
mdc(n_1, n_2) = 1 &\implies ord(g_1 \cdot g_2) = n_1 \cdot n_2
\end{align*}

Vamos contruir elementos de ordem específica, fator de $p-1$ da seguinte forma:
\begin{align*}
b_j &= a^{(p-1)/p_j^{e_j}} \text{ para } j = 1, 2, \dots, m\\
ord(b_j) &| p_j^{e_j}                           &&\text{Por construção}\\
b_j ^ {(p-1)/p_j} &\neq 1 \implies ord(b_j)=p_j^{e_j}
\end{align*}
Dessa forma, quando temos a fatoração total de $p-1$, para cada fator primo $p_i^{e_i}$, podemos encontrar um elemento da ordem do fator e os multiplicar para obter um gerador de ordem $p-1$. 
Já quando temos apenas a fatoração parcial, fazemos o mesmo processo para os fatores $p_i^{e_i}$ conhecidos e para o fator desconhecido $N$, porém aqui não saberemos a ordem exata para este elemento, apenas que está entre $\frac{p-1}{N}$ e $p-1$. 

\noindent\hspace{0.03\linewidth}
\begin{minipage}{.9\linewidth}
\begin{lstlisting}[language=haskell,caption=Algoritmo de Gauss]
highOrderElement :: Factorization -> (Integer, OrderBounds)
highOrderElement f = (g, o) where
    p = defactorize f + 1
    (fs, o) = case f of
        Full fs -> (fs, Exact (p-1))
        Partial fs r -> (r:fs, Bounded (product fs * firstPrimeGT softLimit) (p-1))
    g = foldl (\a b -> mod (a*b) p) 1 bs
    bs = map findB.multiplicity$fs
    findB (f, k) = head
        . filter (\b -> binExp b (div (p-1) f) p /= 1)
        . map (\a -> binExp a (div (p-1) (f^k)) p)
        $ primes
    multiplicity = map (\f -> (head f, length f)).group.sort
    
\end{lstlisting}
\end{minipage}

\subsection{Pohlig-Hellman Generalizado}
O algorítmo de Pohlig-Hellman é utilizado para resolver a equação
$$ a^k = b \text{ em } G = \langle a \rangle\text{, com }|G| = n\text{ e }b \in G$$
$$ n = \prod_{i=1}^{m}p_i^{e_i} \text{ fatoração em primos} $$
Para cada fator primo $p_i$ com multiplicidade $e_i$, nós calculamos
\begin{align*}
    a_i &= a^{n/p_i^{e_i}}\\
    b_i &= b^{n/p_i^{e_i}}\\
        &= a^{k n/p_i^{e_i}} = a_i^k\\
    b_i = a_i^k &\implies b_i \in \langle a_i \rangle            &&\text{Por construção}\\
    ord(a_i) = p_i^{e_i} &\implies b_i = a_i^{k \bmod p_i^{e_i}} &&\text{Pelo teorema de Lagrange}
\end{align*}
Chamemos $ k_i = k \bmod p_i^{e_i} $, podemos então calcular $k_i$ em grupos de ordem $p_i^{e_i}$ ao resolver $a_i^{k_i} \equiv b_i \bmod p_i^{e_i}$, e obter $k$ ao solucionar o sistema
\begin{align*}
    k &= k_1 \mod p_1^{e_1} \\
    k &= k_2 \mod p_2^{e_2} \\
    &\;\;\vdots \\
    k &= k_m \mod p_m^{e_m}
\end{align*}
A implementação calcula $k_i$ para cada fator primo de $n$ com auxílio da função Pohlig-Hellman para grupos de ordem primária em $O(e_i\sqrt{p_i})$, então calcula $k$ utilizando o algorítmo para o teorema Chinês do resto em $O(m)$.
A complexidade da função é $O(\sqrt[4]{n}\ log^2{n})$, já que $m, e \leq \log{n}$ e $p \leq \sqrt{n}$

\noindent\hspace{0.03\linewidth}
\begin{minipage}{.9\linewidth}
\begin{lstlisting}[language=haskell,caption=Pohlig Hellman]
pohligHellman :: Integer -> Integer -> [Integer] -> Maybe Integer
pohligHellman a g f
    | any (\(x, val) -> isNothing x) equations = Nothing
    | otherwise = Just $ crt (map (\(Just x, val) -> (x, val)) equations)
    where
        m = product f + 1
        groupOrder = m - 1
        facts = map (\x -> (head x, toInteger . length $ x)) . group . factorize $ groupOrder
        equations = map (\(q, e) ->
            let
                di = binExp q e m
                a' = binExp a (groupOrder `div` di) m
                g' = binExp g (groupOrder `div` di) m
                ni = pohligPrimePower a' g' (q, e) m
            in (ni, di)) facts
\end{lstlisting}
\end{minipage}


\subsection{Pohlig-Hellman com Ordem Primária}
O algorítmo de Pohlig-Hellman em um grupo com ordem primária, \emph{i.e.} potência de primo, calcula
$$ a^k = b \text{ em } G = \langle a \rangle\text{, com }|G| = p^e\text{ e }b \in G$$
Suponha que tenhamos grupos denotados por
\begin{align*}
    G_0 &= G\\
    G_i &= \lbrace x^{p^i} \vert\ x \in G \rbrace\\
    G_e &= \lbrace x^{p^e} = 1 \vert\ x \in G \rbrace   &&\text{Pela definição de ordem}\\
  |G_i| &= p^{e-i}                                      &&\text{Pelo teorema de Lagrange}
\end{align*}
Façamos uma sequência $b$
\begin{align*}
    b_0 &= b\\
    b_{i+1} &= b_i a^{-k_i p^{i}}
\end{align*}
Definimos $k_i$ como a solução para uma equação utilizando $\pi_i(x) : G_i \rightarrow G_{e-1} |\ x \in G_i$
\begin{align*}
    \pi_i(x) &= x^{p^{e-1-i}}\\
    \pi_0(a)^{k_i} &= \pi_i(b_i)
\end{align*}
Observe que $b_i \in G_i$, e portanto $b_e \in G_e$, então $b_e = 1$
\begin{align*}
    a^{p^{e-1}k_i} &= b_i^{p^{e-1-i}}           &&\text{Definição de $k_i$}\\
    (a^{k_i p^i})^{p^{e-1-i}} &= b_i^{p^{e-1-i}}\\
    b_i = a^{k_ip^{i}} &\implies b_i \in G_i    &&\text{Por construção}\\
\end{align*}
Ao expandir $b_e$, obtemos o inverso de $b_0$
\begin{align*}
    1 &= b_e\\
    &= b_{e-1}a^{-k_{e-1} p^{e-1}}\\
    &= b_{e-2}a^{-\sum_{i=e-2}^{e-1}{k_i p^i}}\\
    &\;\;\vdots\\
    &= b_{0}a^{-\sum_{i=0}^{e-1}{k_i p^i}}\\
    a^{\sum_{i=0}^{e-1}{k_i p^i}} &= b\\
  \text{ então } k &= \sum_{i=0}^{e-1}{k_i p^i}\\
\end{align*}
Assim, resolvemos $e$ equações $\pi(a)^{k_i} = \pi(b_i)$ em $G_{e-1}$, de ordem $p$, utilizando o Baby-Step Giant-Step em $O(\sqrt{p})$, e obtemos $k$ pela equação acima. A complexidade do algorítmo é $O(e\sqrt{p})$

\noindent\hspace{0.03\linewidth}
\begin{minipage}{.9\linewidth}
\begin{lstlisting}[language=haskell,caption=Pohlig Prime Power]
pohligPrimePower :: Integer -> Integer -> (Integer, Integer) -> Integer -> Maybe Integer
pohligPrimePower b a (p, e) m = k where
    tex($\pi$tex) i x = binExp x (p^(e-1-i)) m
    solve b i = babyGiantSteps' (tex($\pi$tex) i b) (tex($\pi$tex) 0 a) m p
    it (i, b, k) = (i+1, b',) <$> solve b' (i+1) where
        b' = mod (b * binExp a (m-1-k*(p^i)) m) m
    k = fmap (sum.map (\(i,_,k) -> k*p^i))
        . sequence
        . take (fromInteger e)
        $ iterate (it=<<) ((0,b,) <$> solve b 0)
\end{lstlisting}
\end{minipage}

\subsection{Baby-Step Giant-Step}
O algorítmo Baby-Step Giant-Step é utilizado para resolver a equação
$$ a^k = b \text{ em } G = \langle a \rangle\text{, com }|G| = n\text{ e }b \in G$$
$$ r = \lceil \sqrt{n} \rceil $$
Dado que $ |G| = n $, então $ a^k = a^{k \mod n} $, e $ 0 \leq k < n $. Assim, re-escrevemos $ k = j r + i $ com $ 0 \leq i, j < r $.\\
Para encontrar $i$ e $j$, o algorítmo calcula duas sequências
\begin{align*}
    x_i &= a^i\\
    y_j &= b a^{-j r} = a^{k - j r}\\
        &= a^{j r + i - j r}\\
        &= a^{i}
\end{align*}
A implementação requer que encontremos $i$ e $j$ tal que $x_i = y_j$. As sequências tem tamanho $r$, são ordenadas em $O(r \log r)$, e buscamos elementos iguais em $O(r)$. A complexidade é $O(r \log r)$

\noindent\hspace{0.03\linewidth}
\begin{minipage}{.9\linewidth}
\begin{lstlisting}[language=haskell,caption=Baby-Steps Giant-Steps]
babyGiantSteps' :: Integer -> Integer -> Integer -> Integer -> Maybe Integer
babyGiantSteps' b a p n = f xs ys where
    r = toInteger . ceiling . sqrt . fromIntegral $ n
    s = invMod (binExp a r p) p
    xs = sortOn fst $ zip (iterate ((`mod`p).(*a)) 1) [0..r-1]
    ys = sortOn fst $ zip (iterate ((`mod`p).(*s)) b) [0..r-1]
    f [] _ = Nothing
    f _ [] = Nothing
    f xxs@((x,i):xs) yys@((y,j):ys)
        | x>y = f xxs ys
        | x<y = f xs yys
        | otherwise = Just $ j*r + i
\end{lstlisting}
\end{minipage}

\end{document}